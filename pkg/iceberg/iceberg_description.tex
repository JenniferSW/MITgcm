%%
%%

Package ``ICEBERG'' is a template for implementing a new package.
It shows how, ideally, the package should interact with the
MITgcm kernel on various levels (initialisation, time-stepping,
post-processing).

Package 'ICEBERG' is a template for iceberg drift and melting.

The package works with an iceberg distribution and treats the icebergs not seperately but as a concentration
within a grid cell.
It is possible to define different size classes for the icebergs which are read in from.
for every size class a single input file is needed.
It is also possible to delay the start of iceberg drift with an own start time.
The parts of reading in input files, implementing size classes and starting the
drift at a certain time step is based on the PTRACERS package.
The parameters introduced in this step are
(data.iceberg)
ICEBERG_numClUsed  ::  number of iceberg size classes used in the simulation (default 1)
ICEBERG_initialFile  ::  File with input data for one size class (add one for each size class)
ICEBERG_Iter0  ::  set to time step when iceberg should be included in the model run (default 0)

(ICEBERG_SIZE.h)
ICEBERG_numCl  ::  number of iceberg classes that are implemented while compiling the model code



The advection of the icebergs is based on the seaice advection from package SEAICE
and GAD_Advection.

